\documentclass{scrartcl}
\usepackage{pgfplots}
\usepackage{graphicx}
\usepackage{pgfplotstable}
\usepackage{csvsimple}
\usepackage[left=1cm, right=1cm, top=1cm, bottom=3cm]{geometry}
\renewcommand{\thesection}{\Roman{section}-}
\renewcommand{\thesubsection}{\Roman{section}-\arabic{subsection}}

\usepackage{titlesec}
\titlelabel{\thetitle\quad}

\begin{document}

	\title{\vspace{-2cm}Compte-rendu de travaux pratiques de chimie}
	\subtitle{Spectrophotométrie}
	\author{Benjamin Loison (MPSI 1)}
	\date{}
	\maketitle

  \setcounter{section}{2}
	\section{Partie expérimentale}
	
		\subsection{Vérification de la loi de Beer-Lambert}
	
			On résume les données expérimentales dans le tableau de données suivant:\\
			
			\begin{tabular}{|c|c|}
					\hline $C$ (en mol/L) & absorbance A \\
					\hline 0.4 & 2.534\\
					\hline 0.2 & 1.227\\
					\hline 0.15 & 0.959\\
					\hline 0.075 & 0.477\\
					\hline 0.0375 & 0.241\\
					\hline
				\end{tabular}\\\\
	
			\begin{tikzpicture}
			\begin{axis}[
				xlabel=$A$,
				ylabel=$C$ (en mol/L)]
			\addplot[color = red, mark = x] coordinates {
				(0.4, 2.534)
				(0.2, 1.224)
				(0.15, 0.959)
				(0.075, 0.477)
				(0.0375, 0.241)
			};
			\end{axis}
		\end{tikzpicture}
		
		On remarque que le nuage de points forment une droite passant par l'origine ainsi la loi de Beer-Lambert est vérifiée. Plus précisément à l'aide d'une régression linéaire on trouve $C = 158.5 * A$.
	
		\subsection{Spectre d'un indicateur coloré acido-basique}
			
				%\begin{tabular}{l|c|c|c}%
				%	\bfseries $\lambda$ (en nm) & \bfseries A (absorbance à pH = 9) & \bfseries A (à pH = 2) & \bfseries A (à pH = 3.66)
				%	\csvreader[head to column names]{data/courbe.csv}{}
				%	{\\\hline\l & \a & \b & \c}
				%\end{tabular}

				\pgfplotstableread[col sep = comma]{data/courbe1.csv}\one
				\pgfplotstableread[col sep = comma]{data/courbe2.csv}\two
				\pgfplotstableread[col sep = comma]{data/courbe3.csv}\three

				\begin{tikzpicture}
					\begin{axis}[
						xlabel=$\lambda$ (en nm),
						ylabel=$A$]
						\addplot[color = red, mark = .] table {\one};
						\addplot[color = green, mark = .] table {\two};
						\addplot[color = blue, mark = .] table {\three};
					\end{axis}
				\end{tikzpicture}
			
				A pH = 9, on remarque un maximum d'absorption de 0.9486 à la longueur d'onde 591 nm qui correspond à une couleur visible bleue. (couleur opposée à 591 nm sur une représentation chromatique).\\
				A pH = 2, on remarque un maximum d'absorption de 0.3104 à la longueur d'onde 437 nm qui correspond à une couleur visible jaune.\\
				On obtient un pH intermédiaire de 3.66, on remarque alors deux maximum d'absorption aux longeurs d'onde 439 nm et 591 nm où l'absorbance est respectivement de 0.1960 et 0.3116.
			
			\subsubsection{Longeurs d'onde des maxima d'absorption}
			

			
			\subsubsection{Point isobestique}
			
				On relève cette intersection non précise des trois courbes à une longueur de 495 nm avec pour absorbance respectives des 3 courbes:\\
				$A_1$ = 0.09260, $A_2$ = 0.09820 et $A_3$ = 0.08500.\\
				Par simple moyenne on obtient $A = 0.0919$.
			
			\subsubsection{Détermination du pKa de HIn/In$^-$}
			
				A partir des relations:\\
				$A_A(\lambda) = [HIn]\epsilon_A(\lambda)L$\\
				$A_B(\lambda) = [In^-]\epsilon_B(\lambda)L$\\\\
				
				On note:
				\begin{equation}
					A_B(\lambda) - A(\lambda) = -A_A(\lambda)
				\end{equation}
				\begin{equation}
					A(\lambda) - A_A(\lambda) = -A_B(\lambda)
				\end{equation}
				
				On obtient alors:\\
				$log\left(\frac{(1)}{(2)}\right) = log\left(\frac{A_A(\lambda)}{A_B(\lambda)}\right) = log\left(\frac{\epsilon_A(\lambda)}{\epsilon_B(\lambda)}\right) + log\left(\frac{[HIn]}{[In^-]}\right)$.\\
				D'où: $log\left(\frac{A_B(\lambda) - A(\lambda)}{A(\lambda) - A_A(\lambda)}\right) = log\left(\frac{\epsilon_A(\lambda)}{\epsilon_B(\lambda)}\right) + pKa - pH$.\\
				Si $\frac{\epsilon_A(\lambda)}{\epsilon_B(\lambda)} \approx 1$ alors $log\left(\frac{\epsilon_A(\lambda)}{\epsilon_B(\lambda)}\right) \approx 0$ donc $pKa = pH + log\left(\frac{A_B(\lambda) - A(\lambda)}{A(\lambda) - A_A(\lambda)}\right)$
			
\end{document}
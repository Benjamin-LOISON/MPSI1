\documentclass{scrartcl}
\usepackage{pgfplots}
\usepackage{graphicx}
\usepackage{gensymb}
\usepackage[left=1cm, right=1cm, top=1cm, bottom=3cm]{geometry}
\usepackage{titlesec}
\setlength{\parindent}{0pt}
\titlelabel{\thetitle\quad}

\begin{document}

	\title{\vspace{-2cm}Compte-rendu de travaux pratiques de chimie}
	\subtitle{Dosages acido-basique de l'eau de Perrier}
	\author{Benjamin LOISON et Alice MILFORD ASSEO (MPSI 1)}
	\date{30 mars 2019}
	\maketitle

  \setcounter{section}{1}
	\section{Exploitation}
		\subsection{Courbe n\degree1}
		
			La réaction de dosage est la suivante: $H_2CO_3 + OH^- \rightleftharpoons HCO_3^- + H_2O$\\
			
			\begin{tabular}{@{}cc@{}}
			
			\begin{tabular}{|c|c|}
			\hline v (en mL) & pH \\
			\hline 0 & 5.34 \\
			\hline 1 & 5.40 \\
			\hline 2 & 5.53 \\
			\hline 3 & 5.64 \\
			\hline 4 & 5.74 \\
			\hline 5 & 5.83 \\
			\hline 6 & 5.91 \\
			\hline 7 & 6.01 \\
			\hline 8 & 6.07 \\
			\hline 9 & 6.16 \\
			\hline 10 & 6.24 \\
			\hline 11 & 6.34 \\
			\hline 12 & 6.39 \\
			\hline 13 & 6.49 \\
			\hline 14 & 6.59 \\
			\hline 15 & 6.72 \\
			\hline 16 & 6.85 \\
			\hline 17 & 7.01 \\
			\hline 18 & 7.24 \\
			\hline 19 & 7.63 \\
			\hline 20 & 8.24 \\
			\hline 21 & 8.67 \\
			\hline 22 & 8.91 \\
			\hline 23 & 9.08 \\
			\hline 24 & 9.22 \\
			\hline 25 & 9.35 \\
			\hline 26 & 9.45 \\
			\hline 27 & 9.55 \\
			\hline 28 & 9.65 \\
			\hline 29 & 9.74 \\
			\hline 30 & 9.82 \\
			\hline
		\end{tabular}

			\begin{tikzpicture}
			\begin{axis}[
				xlabel=$v$ (en mL),
				ylabel=$pH$]
			\addplot[color = red, mark = x] coordinates {
				(0, 5.34)
				(1, 5.4)
				(2, 5.53)
				(3, 5.64)
				(4, 5.74)
				(5, 5.83)
				(6, 5.91)
				(7, 6.01)
				(8, 6.07)
				(9, 6.16)
				(10, 6.24)
				(11, 6.34)
				(12, 6.39)
				(13, 6.49)
				(14, 6.59)
				(15, 6.72)
				(16, 6.85)
				(17, 7.01)
				(18, 7.24)
				(19, 7.63)
				(20, 8.24)
				(21, 8.67)
				(22, 8.91)
				(23, 9.08)
				(24, 9.22)
				(25, 9.35)
				(26, 9.45)
				(27, 9.55)
				(28, 9.65)
				(29, 9.74)
				(30, 9.82)
			};
			\end{axis}
		\end{tikzpicture}		
		\end{tabular}\\
		
		On trouve $V_1$ = 19.5 mL.\\
		A l'équivalence, on a: $[H_2CO_3]V_{Perrier} = [OH^-]V_1$\\
		D'où: $[H_2CO_3] = \frac{[OH^-]V_1}{V_{Perrier}}$\\
		AN: $x = [H_2CO_3] = \frac{0.1 * 19.5 * 10^{-3}}{50 * 10^{-3}} = 3.9 * 10^{-2}$ mol/L.
	
		\subsection{Courbe n\degree2}
			
			La réaction de dosage est la suivante: $HCO_3^- + Cl^- \rightleftharpoons CO_4^{2-} + HCl$
			
			\begin{tabular}{@{}cc@{}}
			
			\begin{tabular}{|c|c|}
			\hline v (en mL) & pH \\
			\hline 0 & 5.35 \\
			\hline 1 & 5.33 \\
			\hline 2 & 5.21 \\
			\hline 3 & 5 \\
			\hline 4 & 4.65 \\
			\hline 5 & 3.25 \\
			\hline 6 & 2.65 \\
			\hline 7 & 2.47 \\
			\hline 8 & 2.36 \\
			\hline 9 & 2.28 \\
			\hline 10 & 2.22 \\
			\hline 11 & 2.17 \\
			\hline 12 & 2.13 \\
			\hline 13 & 2.09 \\
			\hline 14 & 2.05 \\
			\hline 15 & 2.02 \\
			\hline
		\end{tabular}

			\begin{tikzpicture}
			\begin{axis}[
				xlabel=$v$ (en mL),
				ylabel=$pH$]
			\addplot[color = red, mark = x] coordinates {
				(0, 5.35)
				(1, 5.33)
				(2, 5.21)
				(3, 5)
				(4, 4.65)
				(5, 3.25)
				(6, 2.65)
				(7, 2.47)
				(8, 2.36)
				(9, 2.28)
				(10, 2.22)
				(11, 2.17)
				(12, 2.13)
				(13, 2.09)
				(14, 2.05)
				(15, 2.02)
			};
			\end{axis}
		\end{tikzpicture}		
		\end{tabular}\\
		
		On trouve $V_2$ = 4.5 mL.\\
		A l'équivalence, on a: $[HCO_3^-]V_{Perrier} = [Cl^-]V_2$\\
		D'où: $[HCO_3^-] = \frac{[Cl^-]V_2}{V_{Perrier}}$\\
		AN: $y = [HCO_3^-] = \frac{0.1 * 4.5 * 10^{-3}}{50 * 10^{-3}} = 9.0 * 10^{-3} mol/L.$
			
		\subsection{pH de l'eau de Perrier}
			
			On a: pH = pKa + log($\frac{y}{x}$)\\
			AN: pH = 6.3 + log($\frac{9.0 * 10^{-3}}{3.9 * 10^{-2}}$) = 5.66.\\
			Expérimentalement on mesure un pH de 5.34.\\
			On a: $|\frac{pH_{th\acute{e}orique} - pH_{exp\acute{e}rimental}}{pH_{exp\acute{e}rimental}}| = |\frac{5.66 - 5.34}{5.34}| = 5.99 * 10^{-2}$.\\
			On trouve un écart relatif entre la valeur expérimentale et celle théorique de 6 \%.
			
		\subsection{"Raccorder" les deux courbes précédentes}
		
			\begin{tabular}{@{}cc@{}}
			
			\begin{tabular}{|c|c|}
			\hline v (en mL) & pH \\
			\hline 0 & 5.35 \\
			\hline
		\end{tabular}

			\begin{tikzpicture}
			\begin{axis}[
				xlabel=$\leftarrow V_{HCl}$ (en mL) $\hspace{2cm} V_{NaOH}$ (en mL) $\rightarrow$,
				ylabel=$pH$]
			\addplot[color = red, mark = x] table {raccord.txt};
			\end{axis}
		\end{tikzpicture}		
		\end{tabular}\\
			
\end{document}